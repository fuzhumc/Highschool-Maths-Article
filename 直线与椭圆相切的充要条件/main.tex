\documentclass{article}
\usepackage{../mathnotezh}

\hypersetup{
    pdftitle={直线与椭圆相切的充要条件},
    pdfauthor={tbj},
    pdfsubject={Maths},
    pdfcreator={LaTeX},
    pdfproducer={XeLaTeX}
}

\title{直线与椭圆相切的充要条件}
\author{tbj}
\date{}

\begin{document}

\maketitle

对于直线$l: Ax + By + C = 0$与椭圆$\Gamma: \dfrac{x^2}{a^2} + \dfrac{y^2}{b^2} = 1$,欲判定二者是否相切,
常规的办法是联立消去$x$(或$y$),得到一个关于$y$(或$x$)的一元二次方程,然后判定其判别式是否为零.
但这样计算复杂.事实上,我们有如下定理

\begin{thm}
    直线$l: Ax + By + C = 0$与椭圆$\Gamma: \dfrac{x^2}{a^2} + \dfrac{y^2}{b^2} = 1$相切的充分必要条件是
    \[ A^2a^2 + B^2b^2 = C^2 \]
\end{thm}

接下来我们给出三种证法.

\paragraph{法一}

\begin{enumerate}[label=(\arabic*)]
    \item \label{1} 若$l$与$\Gamma$相切,设切点为$P(x_0, y_0)$.
        $\Gamma$在$P$处的切线方程为$\dfrac{x_0x}{a^2} + \dfrac{y_0y}{b^2} = 1$.因此
        \[ \begin{cases}
            \dfrac{x_0}{a^2} = -\dfrac{A}{C} \\
            \dfrac{y_0}{b^2} = -\dfrac{B}{C}
        \end{cases} \]
        解得
        \[ \begin{cases}
            x_0 = -\dfrac{Aa^2}{C} \\
            y_0 = -\dfrac{Bb^2}{C}
        \end{cases} \]
        由$P$在$l$上知$Ax_0 + By_0 + C = 0$,代入知$A^2a^2 + B^2b^2 = C^2$.
    \item \label{2} 若$A^2a^2 + B^2b^2 = C^2$,取点$P(x_0, y_0)$,其中
        \[ \begin{cases}
            x_0 = -\dfrac{Aa^2}{C} \\
            y_0 = -\dfrac{Bb^2}{C}
        \end{cases} \]
        由\[ \dfrac{x_0^2}{a^2} + \dfrac{y_0^2}{b^2} = \dfrac{A^2a^4}{a^2C^2} + \dfrac{B^2b^4}{b^2C^2}
        = \dfrac{A^2a^2 + B^2b^2}{C^2} = 1 \]知点$P$在$\Gamma$上.
        进而$P$处的切线方程为$\dfrac{x_0x}{a^2} + \dfrac{y_0y}{b^2} = 1$.
        即$-\dfrac{A}{C} x - \dfrac{B}{C} y = 1$,亦即$Ax + By + C = 0$.
        因此$l$与$\Gamma$相切.
\end{enumerate}

综合 \ref{1} 与 \ref{2},定理得证.

\begin{rqq}
    证明过程中运用了椭圆上一点处的切线公式$\dfrac{x_0x}{a^2} + \dfrac{y_0y}{b^2} = 1$.
\end{rqq}

\paragraph{法二}

作仿射变换$x' = \dfrac{x}{a}, y' = \dfrac{y}{b}$,则$\Gamma': x'^2 + y'^2 = 1$为单位圆,$l': Aax' + Bby' + C = 0$,
则原点到$l'$的距离为

\[d = \dfrac{|C|}{\sqrt{A^2a^2 + B^2b^2}} \]

因此

\begin{align*}
    l \text{与} \Gamma \text{相切} & \iff l' \text{与} \Gamma' \text{相切} \\
    & \iff d = 1 \\
    & \iff A^2a^2 + B^2b^2 = C^2
\end{align*}

定理得证.

\paragraph{法三}

当$l$过原点时,显然$l$与$\Gamma$相交,且$A^2a^2 + B^2b^2 = C^2$.
当$l$不过原点时,考虑齐次化$1 = -\dfrac{Ax + By}{C}$,代入$\Gamma$的方程得

\[ \dfrac{x^2}{a^2} + \dfrac{y^2}{b^2} = \dfrac{(Ax + By)^2}{C^2} \]

同除$x^2$化简得

\[ (B^2a^2b^2 - a^2C^2) \left( \dfrac{y}{x} \right)^2 + (2ABa^2b^2) \left( \dfrac{y}{x} \right) + (A^2a^2b^2 - b^2C^2) = 0 \]

其判别式为

\begin{align*}
    \Delta & = (2ABa^2b^2)^2 - 4(B^2a^2b^2 - a^2C^2)(A^2a^2b^2 - b^2C^2) \\
    & = 4A^2B^2a^4b^4 - 4(B^2a^2b^2 - a^2C^2)(A^2a^2b^2 - b^2C^2) \\
    & = 4a^2b^2C^2(A^2a^2 + B^2b^2 - C^2)
\end{align*}

故$l$与$\Gamma$相切$\iff \Delta = 0 \iff A^2a^2 + B^2b^2 = C^2$.定理得证.

\end{document}